%%%%%%%%%%%%%%%%%%%%%%%%%%%%%%%%%%%%%%%%%
% Twenty Seconds Resume/CV
% LaTeX Template
% Version 1.0 (14/7/16)
%
% Original author:
% Carmine Spagnuolo (cspagnuolo@unisa.it) with major modifications by 
% Vel (vel@LaTeXTemplates.com) and Harsh (harsh.gadgil@gmail.com)
%
% License:
% The MIT License (see included LICENSE file)
%
%%%%%%%%%%%%%%%%%%%%%%%%%%%%%%%%%%%%%%%%%

% Originally found on https://www.sharelatex.com/templates/cv-or-resume/smart-twenty-seconds-cv

%----------------------------------------------------------------------------------------
%	PACKAGES AND OTHER DOCUMENT CONFIGURATIONS
%----------------------------------------------------------------------------------------

\documentclass[letterpaper]{twentysecondcv} % a4paper for A4


% Command for printing framework bubbles
\newcommand\modelling{ 
~
	\smartdiagram[bubble diagram]{
        \textbf{Machine}\\\textbf{Learning},
        \textbf{Dim}\\\textbf{reduction},%
        \textbf{Logistic} \\\textbf{regression},
        \textbf{Markov}\\\textbf{chains},
        \textbf{Likelihood}\\\textbf{estimation},
        \textbf{Naive}\\\textbf{Bayes},
        \textbf{Conv}\\\textbf{Nets}
    }
    
}

% Command for printing skill overview bubbles
\newcommand\skills{ 
~
	\smartdiagram[bubble diagram]{
        \textbf{Data}\\\textbf{Analysis},
        \textbf{Full Stack}\\\textbf{Web Dev},
        \textbf{Biomedical} \\\textbf{Engineering},%
        \textbf{Deep}\\\textbf{Learning},
        \textbf{Machine}\\\textbf{Learning},
        \textbf{Image}\\\textbf{recognition},
        \textbf{Statistical}\\\textbf{Analysis}
    }
}

% Programming skill bars
\programming{{Pandas $\textbullet$ Keras  $\textbullet$ VTK / 1}, {JS $\textbullet$ SQL $\textbullet$ \large MATLAB / 2}, {C++ $\textbullet$ PHP $\textbullet$ Python / 5}}


% Projects text
\awards{
\awarditems{2017}{EPSRC early career scientist fellowship top 10 finalist  for £1m}\\[1.5mm]
\awarditems{2016}{Journal of Cardiovascular Imaging highly cited research award}\\[1.5mm]
\awarditems{2015}{Imperial College Honorary Lecturer}\\[1.5mm]
\awarditems{2014}{Medical Engineering UK best paper poster}\\[1.5mm]
\awarditems{2012}{Hamlyn Robotics UK best paper poster}\\[1.5mm]
\awarditems{1998}{University of Toronto Scholar award}\\[1.5mm]

}
\talks{
\\[2mm]

\textit{Recent only}\\[1mm]
\talkitems{2018}{Prediction of pacing sites for cardiac pacemakers, IEEE EMBC}{USA}\\[2mm]
\talkitems{2017}{Visualisation and uncertainty in whole-heart modelling, Zuse Maths Inst.}{Berlin} \\[2mm]
\talkitems{2017}{Problem solving using biomedical imaging analysis, EuroSciCon}{UK} \\[2mm] 
\talkitems{2015}{Data analytics and visualisation in interventional cardiac MRI, SCMR}{France}\\[2mm]
\talkitems{2015}{Quantification algorithms to assess heart attacks from imaging, Hounsfield Lectures}{UK}\\[2mm]
}

\teaching{
\\[1mm]
\textit{Supervised theses (recent)}\\[1mm]
\teachingitems{2017}{Model-based image analysis of cardiac wall thickness using measurements from CT} \\[1mm]
\teachingitems{2017}{Image processing algorithms for measuring wall thickness in CMR} \\[1mm]
\teachingitems{2016}{Tissue contact force sensing in uni and multi-directional catheters} \\[1mm] 
\teachingitems{2016}{Robotic ablation catheter - experimentations on precision and control} \\[1mm]
\teachingitems{2016}{Validation of an adapted Cosserat rod model for contact force estimation} \\[1mm]
\teachingitems{2016}{Ablation path trajectories of a robotic catheter inside a heart phantom} \\[1mm]
\begin{footnotesize}
\textit{Years 2012 - 2016, 12 theses were supervised}\\
\end{footnotesize}
}

\editorial{
\\[1mm]
\editorialitems{J. of Healthcare Engineering}{Editorial board}
\editorialitems{Asian J. of Info. Technology}{Editorial board}
\editorialitems{Trans. Pattern Analy. \& Mach. Intel.}{Reviewer}
\editorialitems{Sensors}{Reviewer}
\editorialitems{IEEE Trans. Medical Imaging}{Reviewer}
}


%----------------------------------------------------------------------------------------
%	 PERSONAL INFORMATION
%----------------------------------------------------------------------------------------
% If you don't need one or more of the below, just remove the content leaving the command, e.g. \cvnumberphone{}

\cvname{Rashed Karim} % Your name
\cvjobtitle{ Image Data Scientist } % Job
% title/career

\cvlinkedin{/in/karimphd}
\cvgithub{drkarim}
\cvnumberphone{(44) 79 69 77 03 72} % Phone number
\cvsite{bit.ly/drkarim} % Personal website
\cvscholar{bit.ly/scholarX} % scholar
\cvmail{rashed.karim@gmail.com} % Email address

%----------------------------------------------------------------------------------------



\begin{document}

\makeprofile % Print the sidebar

%----------------------------------------------------------------------------------------
%	 EDUCATION
%----------------------------------------------------------------------------------------
\section{Education}

\begin{twenty} % Environment for a list with descriptions
	\twentyitem
    	{2005 - 2009}
        {}
        {PhD. Computer Science }
        {\href{http://www.doc.ic.ac.uk/}{Imperial College London}}
        {}
        {}
	\twentyitem
    	{2004 - 2005}
		{}
        {MSc. Advanced Computing \textnormal{(Distinction)}}
        {\href{http://www.qmul.ac.uk/}{Queen Mary, U. of London}}
        {}
        {}
    	\twentyitem
    	{1998 - 2002}
		{}
        {BSc. Computing \& Mathematics}
        {\href{http://www.utoronto.ca/}{University of Toronto, Canada}}
        {}

	%\twentyitem{<dates>}{<title>}{<organization>}{<location>}{<description>}
\end{twenty}


%----------------------------------------------------------------------------------------
%	 EXPERIENCE
%----------------------------------------------------------------------------------------

\section{Experience}

\begin{twenty} % Environment for a list with descriptions
\twentyitem
    	{Sep 2010 -}
		{Present}
        {Research Fellow}
        {\href{https://kclpure.kcl.ac.uk/portal/rashed.karim.html}{King's College London}}
        {}
        {\begin{itemize}
       \item Making predictions from 3D imaging data using ML and AI. Quantitative analysis of data. Gaining over 10 years' experience in image data analytics. 
\item Published 41 research papers in leading journals and conferences in last 7 years. An authority on cardiac image analysis
\item Co-chaired and organised 3 international data challenges at premier medical imaging conferences
\item Designed software for NHS research units in Leeds \& London. 
\item Commercialisation of technology used for implanting cardiac pacemakers with Siemens Healthcare
\item Long track record of project supervision: 42 individual projects and 18 BSc student projects (13/18 obtained a 1st grade). 
       
        \end{itemize}}
        \\
	\twentyitem
    	{2007 - }
		{Present}
        {Full Stack Development Consulting}
        {\href{http://www.desilvatutors.co.uk/}{Karim Consulting LTD}}
        {}
        {
        {
    \begin{itemize}
\item Architect and full-stack developer of a London agency's cloud-based CRM system in PHP/MySQL ($\approx$30,000 lines of code) 
		\item The CRM software now runs most of the day-to-day operations of its business. As of Oct. 2018, there are 30 users, processing over 1000 claims and invoices each month. 
    \end{itemize}}
        }
    \\   
    \twentyitem
   		{Sep 2009 -}
		{Dec 2010}
        {Research Associate}
        {\href{http://www.imperial.ac.uk}{Imperial College London}}
        {}
        {
        {\begin{itemize}
        \item Designed software to recognise dead tissue from images, work was published as scientific article and cited 73 times on Google Scholar
    \end{itemize}}
        }
     \\
     \twentyitem
   		{Sep 2007 -}
		{Sep 2008}
        {Software Developer Intern}
        {\href{http://www.ey.com/}{Ernst and Young London}}
        {}
        {
        \begin{itemize}
        \item Member of infrastructure team, building plug-ins and deploying for EY global. Continued employment after internship. Declined offer of further employment to complete PhD.
        \end{itemize}
     	}   
    	\\ 
      \twentyitem
   		{2003 - 04}
		{}
        {Lecturer in Mathematics}
        {\href{http://www.primeasia.edu.bd/}{Primeasia university, Bangladesh}}
        {
        \begin{itemize}
        \item 1st and 2nd year university-level calculus. Youngest lecturer in university. Resigned from post to pursue further study. 
 
        \end{itemize}
        }{}
	\\ 
      \twentyitem
   		{2002 - 03}
		{}
        {Java developer}
        {\href{https://en.wikipedia.org/wiki/BEA_Systems}{BEA Weblogic Dubai}}
		{
		\begin{itemize}
		\item JSP and Servlets development. Sun Java certified programmer
		\end{itemize}		
		}{}
	\\ 
      \twentyitem
   		{2001 - 02}
		{}
        {Part-time Tutor in Microprocessor Systems}
        {\href{https://www.utoronto.ca}{University of Toronto}}
		{}{}
        
        
	%\twentyitem{<dates>}{<title>}{<location>}{<description>}
\end{twenty}

\section{Publications}
\begin{twenty}
\twentyitemshort{Journals}{28 publications, 8 as first author, 4 as second}
\twentyitemshort{Conference}{22 proceedings, 9 as lead author}
\twentyitemshort{Abstracts}{24 clinical abstracts, 14 as first, second or lead author}
\twentyitemshort{\textit{Summary}}{Published 50 articles in 10 years. 23 highly influential citations}
\twentyitemshort{Full list}{\bluehyperlink{http://bit.ly/scholarX}{Google scholar}, \bluehyperlink{http://bit.ly/semanticrk}{Semantic Scholar}, \bluehyperlink{http://bit.ly/pubmedrk}{PubMed}, \bluehyperlink{http://bit.ly/researchgaterk}{ResearchGate}, \bluehyperlink{http://bit.ly/orcidrk}{ORCID}, \bluehyperlink{http://bit.ly/publonsrk}{Publons}}
\end{twenty}

\newpage
\makesecondprofile  % Print the sidebar % Print the sidebar
\section{Projects in industry and academia}
\begin{twenty}
	\twentyitem
    	{2016 - 18}
		{}
        {Surgical guidance for pacemaker implantation}
        {\href{https://www.healthcare.siemens.co.uk/about}{Siemens}}
        {}
        {
		Designed and wrote software to obtain live cardiac tissue information from imaging and projection onto a map. Also built a decision support system for surgeons to use this information and implant pacemakers in the patient's heart. First built as a prototype with Siemens Healthcare and later translated into a commercial system. Work is set to appear in the press in Nov. 2018.}
		\\
	\twentyitem
    	{2018}
		{}
        {Neural networks to learn radiologist's annotations}
        {\href{http://www.medcai.co.uk}{MedcAI UK}}
        {
		Supervised a summer student to design a neural network that learns manual image annotations of radiologists. The network was trained on over 10,000 annotations. It took part in an international challenge held in MICCAI 2018, Granada, Spain, and came as one of top 5 algorithms. An online portal where this neural network can be run without any knowledge or experience of AI has been developed.}
		\\
	\twentyitem
    	{2017}
		{}
        {Augmented reality in museums}
        {\href{https://www.kcl.ac.uk/gordon/index.aspx}{Gordon museum of Pathology}}
        {
		Wrote full stack software of an augmented reality (AR) platform. The AR app runs on hand-held devices and is able to recognise 34 specimens in the museum and displays an AR layer with additional content. The system is in use regularly by visitors helping them explore in a completely different way. }
		\\ 

	\twentyitem
	{2013 - 18}
		{}
        {Curating cardiac imaging data}
        {\href{https://www.nihr.ac.uk/}{National Institute for Health Research}}
        {Undertook a 5-year effort with international collaborators to collect well-curated MRI and CT imaging data. On these datasets, benchmarks were established for algorithm accuracy. Designed a system for algorithm evaluation and made the data open source. I co-chaired meetings held in  Nice (2012), Barcelona (2013) and Athens (2016) to discuss outcomes of these initiatives.  }
	\\
	\twentyitem
	{2016}
		{}
        {Responsive design and RESTful APIs}
        {\href{http://www.desilvatutors.co.uk}{De Silva Tutors UK}}
        {
Re-designed website with a new responsive design. Also re-wrote a number of its API in PHP, JS and SQL following a new RESTful architecture. Wrote new modules to integrate Google maps search for its clients and tutors. A new CV search implemented with with Elastic search engine. This boutique tutoring agency is one of the largest in London with over 100 tutors and 1000 clients in its books.}
	\\
	\twentyitem
	{2013}
		{}
        {Dimensionality reduction of 3D data visualistions}
        {\href{http://bit.ly/2Pjhm9x}{Wellcome Trust}}
        {
An ambitious project of my research lab to create the first 2D flat map of the heart, allowing an instant single shot view of 3D cardiac data. As lead in this project, a practical solution was engineered using spatial dimensionality reduction technique, implemented in C++ and demonstrated with live surgical data feed for clinical use. The technique, now pubished in \textit{Karim et al. Computerized Med. Img. Graphics, 2014}, has highly influenced research work in UPF Barcelona}

	
\end{twenty}

\section{Community engagements}

\begin{twenty}
\community{2018}{Science exhibition}{Sutton Grammar School}
\community{2016}{Science careers for school children}{Sutton Grammar School}
\community{2015}{Science careers for school children}{Sutton High School} 
\community{2014}{Biomedical engineering careers}{Sutton Grammar School}
\community{2011}{Enganging the public in scientific research}{London ExCeL Centre} 
\end{twenty}

\newpage
\newgeometry{left=1cm,top=0.5cm,right=1cm,bottom=0.5cm}
\renewcommand{\theenumi}{J\arabic{enumi}}
\section{Publications list}
($\dagger$ - as first author, $\ddagger$ - as second author, * - as last author, IMP - Journal impact factor)\\[2mm]
\justblue{{\large{\textbf{Journal articles:}}}}
\begin{etaremune}
\item $\dagger$Algorithms for left atrial wall segmentation and thickness - evaluation on an open-source CT and MRI image database, \textbf{Medical Image Analysis}, 2018
\item A work flow to build and validate patient specific left atrium electrophysiology models from catheter measurements, \textbf{Medical Image Analysis}, 2018
\item The optimization of post-ablation atrial scar imaging: a cross-over study, \textbf{J. of Cardiovascular Magnetic Resonance}, 2018
\item The reproducibility of late gadolinium enhanced imaging of post-ablation atrial scar: a cross-over study, \textbf{J. of Cardiovascular Magnetic Resonance}, 2018
\item $\dagger$*Left atrial voltage, circulating biomarkers of fibrosis, and atrial fibrillation ablation. A prospective cohort study, \textbf{PLOS One} (\textit{accepted}), (* - joint first authorship) (IMP = 2.9)
\item Standardised Unfold Mapping: A technique to permit left atrial regional data display and analysis, \textbf{J. of Interventional Cardiac Eletrophysiology}  50(1), 2017 (IMP=1.5)
\item A planning and guidance platform for cardiac resynchronisation therapy, In \textbf{IEEE Transactions Medical Imaging} 36(11), 2017 (IMP=3.9)
\item Real time X-MRI guided left ventricular lead implantation for targeted delivery of cardiac resynchronization therapy, in \textbf{JACC: Clinical Electrophysiology}, \textit{(in press)}
\item $\dagger$*Intra-cardiac and peripheral levels of biochemical markers of fibrosis in patients undergoing catheter ablation for atrial fibrillation, \textbf{Europace} 19(12), 2017 (* - joint first authorship) (IMP=3.7)
\item ECG imaging of ventricular tachycardia: evaluation against simultaneous non-contact mapping and CMR derived grey zone. \textbf{Medical \& Biological Engineering \& Computing} 55(6), 2017 (IMP=1.8)
\item Biophysical modelling predicts ventricular tachycardia inducibility and circuit morphology:  A combined clinical validation and computer modelling approach, \textbf{J. of Cardiovascular Electrophysiology} 27(7), 2016 (IMP=3.1)
\item $\dagger$Evaluation of state-of-the-art segmentation algorithms for left Ventricle infarct from late Gadolinium enhancement MR images, \textbf{Medical Image Analysis}, Vol. 30. pp 95-107, 2016. (IMP=4.5)
\item $\ddagger$3-DOF MR-Compatible Multi-Segment Cardiac Catheter Steering Mechanism \textbf{IEEE Transactions on Biomedical Engineering} 63(11), 2016. (IMP=2.3)
\item $\ddagger$A Randomised Prospective Mechanistic CMR Study Correlating Catheter Stability, Late Gadolinium Enhancement and 3-Year Clinical Outcomes in Robotically-Assisted versus Standard Catheter Ablation.  \textbf{Europace} 17(8), 2015. (IMP=3.7)
\item The Effect of Contact Force in Atrial RF Ablation: Electroanatomical, CMR and Histological Assessment in a Chronic Porcine Model, \textbf{JACC Clinical Electrophysiology} 1(5), 2015. 
\item Benchmark for algorithms segmenting the left atrium from 3D CT and MRI datasets. In \textbf{IEEE Transactions on Medical Imaging}, 2015. (IMP=3.4)
\item Interventional CMR in Electrophysiology - Advances towards clinical translation. \textbf{Circulation: Arrhythmia \& Electrophysiology} 8(1), 2015. (IMP=4.5)
\item Repeat Left Atrial Catheter Ablation: Cardiac Magnetic Resonance Prediction of Endocardial Voltage and Gaps in Ablation Lesion Sets. \textbf{Circulation: Arrhythmia \& Electropysiology} 8(2), 2015. (IMP=4.5)
\item $\dagger$A method to standardise quantification of left atrial scar from delayed-enhancement MR images. \textbf{IEEE Translational Engineering in Health and Medicine} Vol. 2, 2014. (IMP=1.0)
\item $\dagger$Surface Flattening of the Human Left Atrium and Proof-of-Concept Clinical Applications. \textbf{Computerized Medical Imaging and Graphics} 38(4), 2014. (IMP=1.6)
\item $\dagger^*$Quantitative Magnetic Resonance Imaging Analysis of the Relationship between Contact Force and Left Atrial Scar Formation after Catheter Ablation of Atrial Fibrillation. \textbf{J. of Cardiovascular Electrophysiology}, 2014 (* - joint first authorship). (IMP=3.1)
\item A novel skeleton based quantification and 3D volumetric visualization of left atrium fibrosis using Late Gadolinium Enhancement Magnetic Resonance Imaging. \textbf{IEEE Transactions in Medical Imaging}. 2014. (IMP=3.4)
\item Multimodality imaging for catheter ablation of atrial fibrillation. Is it still necessary? \textbf{The International Journal of Cardiology} 175(3), 2014. (IMP=4.0)
\item Cardiac magnetic resonance and electroanatomical mapping of acute and chronic atrial ablation injury. a histological validation study. \textbf{European Heart Journal} 35(22), 2014. (IMP=15.0)
\item  $\dagger$Evaluation of current algorithms for segmentation of scar tissue from late Gadolinium enhancement cardiovascular MR of the left atrium. \textbf{J. of Cardiovascular Magnetic Resonance}. 2013. (IMP=4.7)
\item Automated analysis of atrial late gadolinium enhancement imaging correlates with endocardial voltage and clinical outcomes: a two-center study. \textbf{Heart Rhythm Journal} Vol. 10(8), 2013. (IMP=5.0)
\item Native T1 mapping in differentiation of normal myocardium from diffuse disease in hypertrophic and dilative cardiomyopathy. \textbf{In JACC: Cardiovascular Imaging} 2013. (IMP=6.7)
\item $\ddagger$Acute Pulmonary Vein Isolation Is Achieved by a Combination of Reversible and Irreversible Atrial Injury \textbf{Circulation: Arrhythmia \& Electrophysiology}, 2012. (IMP=4.5)

\end{etaremune}
\renewcommand{\theenumi}{C\arabic{enumi}}

\justblue{{\large{\textbf{Conference proceedings:}}}}
\begin{etaremune} 
\item *Convolutional neural networks for segmentation of the left atrium from gadolinium enhancement MRI images, \textbf{Proceedings of MICCAI-STACOM, Granada, Spain}, 2018 
\item $\dagger$Image data analysis for quantifying scar tissue transmurality in cardiac resynchronisation therapy, \textbf{Proceedings of the IEEE Eng. in Med. and Bio., Hawaii, USA}, 2018
\item $\ddagger$Cardiac NET: Segmentation of the left atrium using Segmentation of left atrium and proximal pulmonary veins from MRI using multi-view convolution neural networks. \textbf{Proceedings of MICCAI, Toronto, Canada}, 2017
\item $\ddagger$A platform for quantifying atrial structure remodelling, \textbf{Proceedings of Computers in Cardiology, Germany}, 2017
\item $\dagger$Segmentation Challenge on the Quantification of Left Atrial Wall Thickness. \textbf{Proceedings of MICCAI-STACOM, Athens, Greece}, 2016
\item $\dagger$Left Atrial Segmentation from 3D Respiratory- and ECG-gated Magnetic Resonance Angiography, \textbf{Proceedings of FIMH Workshop, Maastricht, The Netherlands}, 2015 
	\item $\ddagger$Tension sensing for a linear actuated catheter robot, \textbf{Proceedings of Intel. Robotics and Appl. meeting}, 2015
	\item Interactive visualization for scar transmurality in cardiac resynchronization therapy, \textbf{Proceedings of SPIE Medical Imaging}, San Diego, USA, 2016. 
	\item $\ddagger$Statistical Model of Paroxysmal Atrial Fibrillation Catheter Ablation Targets for Pulmonary Vein Isolation, \textbf{Proceedings MICCAI-STACOM}, 2015
	\item $\ddagger$Catheter contact force estimation from shape detection using a real-time Cosserat rod model, \textbf{Proceedings of IEEE Intelligent Robots and Systems,} 2015.
	\item Left Atrial Segmentation Challenge: A Unified Benchmarking Framework, \textbf{Proceedings of MICCAI-STACOM}, 2014
	\item $\dagger$Infarct Segmentation Challenge on Delayed Enhancement MRI of the Left Ventricle, \textbf{Proceedings of MICCAI-STACOM, Nice, France}, 2012 
	\item $\dagger$Infarct Segmentation of the Left Ventricle Using Graph-Cuts, \textbf{Proceedings of MICCAI-STACOM), Nice, France}, 2012
	\item $\ddagger$Cardiac Unfold: A Novel Technique for Image-Guided Cardiac Catheterization Procedures,\textbf{Proceedings of Information Processing in Computer-Assisted Interventions (IPCAI)}, 2012
	\item $\dagger$Validation of a Novel Method for the Automatic Segmentation of Left Atrial Scar from Delayed-Enhancement Magnetic, \textbf{Proceedings of MICCAI-STACOM,}2011
	\item $\dagger$Automatic Segmentation of Left Atrial Scar from Delayed-Enhancement Magnetic Resonance Imaging, \textbf{Proceedings of Functional Imaging and Modeling of the Heart (FIMH) 2011}
	\item $\dagger$Mapping Contact Force during Catheter Ablation for the Treatment of Atrial Fibrillation, \textbf{Proceedings of Functional Imaging and Modeling of the Heart (FIMH) 2011}, Lecture Notes in Computer Science,  Volume 6666, pp 302-303. 
	\item $\dagger$Automatic Segmentation of Left Atrial Geometry from Contrast-Enhanced MRI using a Probabilistic Atlas, \textbf{In Proceedings of MICCAI-STACOM}, 2010
	\item $\dagger$Left atrium pulmonary veins: segmentation and quantification for planning atrial fibrillation ablaton. \textbf{Proceedings SPIE Medical Imaging, 2009} 
	\item $\dagger$Automatic extraction of the left atrial anatomy from MR for atrial fibrillation ablation, \textbf{Proceedings of IEEE International Symposium on Biomedical Imaging (ISBI)}, Boston, USA, 2009. 
	\item $\dagger$Left Atrium Segmentation for Atrial Fibrillation Ablation, \textbf{Proceedings of SPIE Medical Imaging, San Diego, 2008}. 
	\item $\dagger$Automatic Segmentation of the Left Atrium. \textbf{Proceedings of Medical Image Understanding and Analysis Conference (MIUA)}, Abersytwyth, Wales, 2007. 
\end{etaremune}

\justblue{{\large{\textbf{Abstract \& short papers:}}}}
\renewcommand{\theenumi}{A\arabic{enumi}}
\begin{etaremune}
\item *Validating scar quantification for guiding cardiac resynchronisation therapy, in \textbf{Institute of Physics and Engineering in Medicine MEIBioeng} meeting, 2017
\item *Model-based image analysis of left atrial wall thickness using direct measurements from CT, in \textbf{Institute of Physics and Engineering in Medicine MEIBioeng} meeting, 2017  
\item $\ddagger$Motion correction using hierarchical local affine registration improves image quality and myocardial scar characterisation from T1 maps acquired with MOLLI. In \textbf{Society of Cardiovascular Magnetic Resonance (SCMR) meeting} meeting, 2013
\item $\ddagger$Simultaneous non-Contact mapping fused with CMR derived grey zone to explore the relationship with ventricular tachycardia substrate in ischaemic cardiomyopathy, In \textbf{Society of Cardiovascular Magnetic Resonance (SCMR)} meeting 2013 
\item $\ddagger$Quantitative magnetic resonance imaging analysis of the relationship between contact force and left atrial scar formation after catheter ablation of atrial fibrillation in late-breaking clinical trials at \textbf{Heart Rhythm Society} meeting, 2013 
\item $\dagger$Left atrium surface flattening for assisting guidance in catheter ablation procedures, at \textbf{Hamlyn Symposium on Medical Robotics}, 2012
\item $\ddagger$MR-Compatible autonomous catheterization Robot with unfolded navigational maps (best poster prize), at \textbf{Hamlyn Symposium on Medical Robotics}, 2012
\item Acute Atrial Ablation Injury Is Better Visualised By Late Gadolinium Enhancement Than T2-weighted Magnetic Resonance Imaging, at \textbf{Heart Rhythm Society}, 2012
\item Late Gadolinium Enhancement Magnetic Resonance Imaging Prediction Of Gaps In Atrial Ablation Lesions, at \textbf{Heart Rhythm Society}, 2012  
\item $\dagger$An automatic segmentation algorithm for improved visualization of atrial ablation lesions using magnetic resonance imaging. In  \textbf{Society of Cardiovascular Magnetic Resonance} (SCMR) meeting, 2011
\item $\dagger$Evaluation of a rapid quantification algorithm for delayed enhancement mri following left atrial ablation. In \textbf{Heart Rhythm Society}, 2011
\item $\dagger$Magnetic resonance imaging analysis of tissue-contact force following catheter ablation for paroxysmal atrial fibrillation. \textbf{Heart Rhythm Society} meeting, 2011 
\item Novel dual inversion recovery pre-pulse imaging technique improves post ablation cardiac MR scar visualization. \textbf{Heart Rhythm Society} meeting, 2012
\item Automated analysis of atrial delayed-enhancement cardiac mri correlates with endocardial voltage, In \textbf{Heart Rhythm society} 2011 
\item Acute atrial ablation injury is better visualised by late gadolinium enhancement than t2-weighted magnetic resonance imaging. In \textbf{Heart Rhythm Society} meeting, 2012
\item $\ddagger$Interstitial oedema, delayed enhancement and recurrences following catheter ablation in paroxysmal AF: how are they related? In \textbf{Heart Rhythm Society} meeting, 2011
\item $\ddagger$Assessment of left atrial injury by cardiac MR: a randomised prospective comparison of robotic versus manual AF ablation. In \textbf{Heart Rhythm Society} meeting, 2011
\item Acute pulmonary vein isolation lesions consist of interstitial oedema and tissue necrosis: possible mechanism of pulmonary vein reconnection. \textbf{Society of Cardiovascular Magnetic Resonance }(SCMR) meeting, 2011
\item A novel technique to display tissue contact force in the left atrium following catheter ablation for paroxysmal atrial fibrillation using a force sensing ablation catheter. Abstract only. \textbf{European Cardiac Arrhythmia Society} meeting, 2011
\item Novel pilot data - Cardiac MR Imaging Post Catheter Ablation: Does T2 and DE ratios matter in predicting clinical outcome? \textbf{Proceedings of International Society for Magnetic Resonance in Medicine (ISMRM)} meeting, 2011
\item $\ddagger$Correlation of left-atrial endocardial voltage with ablation-scar detected with DE-MRI. In \textbf{European Cardiology and Arrhythmia Society (ECAS)} meeting, 2011
\item $\ddagger$Left atrial function correlates with post-ablation scar detected by DE-MRI. Abstract only.\textbf{European Cardiology and Arrhythmia Society (ECAS)} meeting, 2011
\item $\ddagger$Automated analysis of atrial ablation scar using delayed enhanced cardiac magnetic resonance imaging. In \textbf{British Cardiovascular Society Annual Conference}, 2011
\item $\ddagger$Applicability of pre segmentation of MRA data to generate 3-dimensional anatomical models of the left atrium and pulmonary veins for planning and facilitation of image guided ablation procedures. In \textbf{European Society of Cardiac Radiology}, 2010 


\end{etaremune}

\end{document}
